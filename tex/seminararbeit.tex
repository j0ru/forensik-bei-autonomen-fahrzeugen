\documentclass[conference,compsoc,final,a4paper]{IEEEtran}
\usepackage[utf8]{inputenx}

%% Bitte legen Sie hier den Titel und den Autor der Arbeit fest
\newcommand{\autoren}[0]{Gleumes, Folke Henning}
\newcommand{\dokumententitel}[0]{Wird autonomes Fahren die forensische Untersuchung erschweren?}

% Hie muss normalerweise nichts angepasst werden
\usepackage[pdftex]{graphicx}
\graphicspath{{img/}}
\DeclareGraphicsExtensions{.pdf,.jpeg,.jpg,.png}
\usepackage[cmex10]{amsmath}
\usepackage{algorithmic}
\usepackage{array}
\usepackage{dblfloatfix}
\usepackage{url}
\usepackage[autostyle=true,german=quotes]{csquotes}
\usepackage[backend=biber,
            sorting=none,   % Keine Sortierung
            doi=true,       % DOI anzeigen
            isbn=false,     % ISBN nicht anzeigen
            url=true,       % URLs anzeigen
            maxnames=6,     % Ab 6 Autoren et al. verwenden
            minnames=1,     % und nur den ersten Autor angeben
            style=ieee,]{biblatex}
\usepackage{booktabs}
\usepackage{xcolor}
\usepackage{listings}             % Source Code listings
\usepackage[printonlyused]{acronym}
\usepackage{fancyvrb}
\usepackage{tocloft} % Schönere Inhaltsverzeichnisse

% Farben definieren
\definecolor{linkblue}{RGB}{0, 0, 100}
\definecolor{linkblack}{RGB}{0, 0, 0}
\definecolor{darkgreen}{RGB}{14, 144, 102}
\definecolor{darkblue}{RGB}{0,0,168}
\definecolor{darkred}{RGB}{128,0,0}
\definecolor{comment}{RGB}{63, 127, 95}
\definecolor{javadoccomment}{RGB}{63, 95, 191}
\definecolor{keyword}{RGB}{108, 0, 67}
\definecolor{type}{RGB}{0, 0, 0}
\definecolor{method}{RGB}{0, 0, 0}
\definecolor{variable}{RGB}{0, 0, 0}
\definecolor{literal}{RGB}{31,0, 255}
\definecolor{operator}{RGB}{0, 0, 0}

\usepackage[ngerman]{babel}

\DefineBibliographyStrings{ngerman}{
    andothers = {{et al\adddot}},  % Immer et al. sagen, auch bei Deutsch als Sprache
}
\usepackage[
      unicode=true,
      hypertexnames=false,
      colorlinks=true,
      colorlinks=false,
      linkcolor=darkblue,
      citecolor=darkblue,
      urlcolor=darkblue,
      pdftex
   ]{hyperref}
%	 \PrerenderUnicode{ü}


% Einstellungen für Quelltexte
\lstset{
    xleftmargin=0.1cm,
    basicstyle=\scriptsize\ttfamily,
    keywordstyle=\color{keyword},
    identifierstyle=\color{variable},
    commentstyle=\color{comment},
    stringstyle=\color{literal},
    tabsize=2,
    lineskip={2pt},
    columns=flexible,
    inputencoding=utf8,
    captionpos=b,
    breakautoindent=true,
    breakindent=2em,
    breaklines=true,
    prebreak=,
    postbreak=,
    numbers=none,
    numberstyle=\tiny,
    showspaces=false,      % Keine Leerzeichensymbole
    showtabs=false,        % Keine Tabsymbole
    showstringspaces=false,% Leerzeichen in Strings
    morecomment=[s][\color{javadoccomment}]{/**}{*/},
    literate={Ö}{{\"O}}1 {Ä}{{\"A}}1 {Ü}{{\"U}}1 {ß}{{\ss}}2 {ü}{{\"u}}1 {ä}{{\"a}}1 {ö}{{\"o}}1
}

\hypersetup{
    pdftitle={\dokumententitel},
    pdfauthor={\autoren},
    pdfdisplaydoctitle=true,
    hidelinks
}

% Makros für typographisch korrekte Abkürzungen
\newcommand{\zb}[0]{z.\,B.}
\newcommand{\dahe}[0]{d.\,h.}
\newcommand{\ua}[0]{u.\,a.}

% Wo liegt Sourcecode?
\newcommand{\srcloc}{src/}

% Literatur einbinden
\addbibresource{literatur.bib}
 % Weitere Einstellungen aus einer anderen Datei lesen

\begin{document}

% Titel des Dokuments
\title{\dokumententitel}

% Namen der Autoren
\author{
  \IEEEauthorblockN{\autoren}
  \IEEEauthorblockA{
    Hochschule Mannheim\\
    Fakultät für Informatik\\
    Paul-Wittsack-Str. 10,
    68163 Mannheim
    }
}

% Titel erzeugen
\maketitle
\thispagestyle{plain}
\pagestyle{plain}

\begin{abstract}
\end{abstract}

% Inhaltsverzeichnis erzeugen
{\small\tableofcontents}

\section{Einleitung}

Seit einigen Jahren befindet sich die Automobilindustrie im Umbruch. Nicht nur steht der Wechsel zu Strom betriebenen Kraftfahrzeugen an,
sondern auch der Wechsel zu immer mehr Computergestützten Fahrsystemen, bis hin zum komplett autonomen Fahrzeug.
Um diesen Umschwung zu ermöglichen, müssen immer mehr Daten erhoben werden und komplexere Systeme zum Auswerten dieser Daten geschaffen werden.
Nach Aussagen von Intel aus dem Jahr 2016 könnten Daten von bis zu 4 Terrabyte pro Tag generiert werden.~\cite{Nelson2016}
Die Firma Tuxera schätzte 2021 das ein durchschnittlicher US-amerikanischer Verbraucher zwischen 380 und 5100 TB pro Jahr generieren könnte.~\cite{Wright2021}
In dieser Arbeit soll die Frage geklärt werden, ob und wie diese Änderung die forensische Auswertung von Fahrzeugen erschweren könnte.

\section{Automatisiertes/Autonomes Fahren}

% Eingehen auf relevante Punkte für die Forensik (z.B. Hände am Lenkrad)

Für den Begriff des autonomen oder automatisierten Fahrens gibt es mehrere Definitionen, die sich jedoch im Kern gleichen.
2013 definierte die \ac{NHTSA} folgende 5 Stufen\cite{NHTSA2013}:
\begin{enumerate}
  \setcounter{enumi}{-1}
  \item \emph{No-Automation:} Der Fahrer hat volle Kontrolle über das Fahrzeug. Dies gillt auch wenn das Fahrzeug über Warnsysteme, wie eine Kollisionswarnung verfügt. Sekundäre Systeme wie Scheibenwischer, Blinklichter, Beleuchtung gelten ebenfalls als Level 0.
  \item \emph{Function-specific Automation:} Die fahrende Person kann teilweise die Kontrolle über einzelne System dem Fahrzeug überlassen. Beispiele für solche Systeme sind das \ac{abs}, welches die ultimative Kontrolle beim Fahrer belässt, aber in den Bremsprozess eingreift und der Spurhalteassistent, welcher nur leicht in den Lenkprozess eingreift, aber jederzeit von der fahrenden Person überschrieben werden kann.
  \item \emph{Combined Function Automation:} Ab dieser Stufe können auch die primären Funktionen vollständig vom Fahrzeug übernommen werden, allerdings muss die fahrzeugführende Person jeder Zeit bereit sein in das Fahrgeschehen einzugreifen. Um dies sicherzustellen, gibt es zum Beispiel Distanzsensoren am Lenkrad, welche messen ob der Fahrer die Hände in der Nähe des Lenkrads hat. %TODO: quelle
  \item \emph{Limited Self-Driving Automation:} Der/Die Fahrer:in kann zeitweise, unter den korrekten Bedingungen, wie z.B. eine Autobahn bei guter Sichtbarkeit, die komplette Kontrolle über das System abgeben. Sollte sich ein Hindernis ankündigen, dass nicht von dem autonomen System übernommen werden kann, wird die fahrende Person benachrichtigt und hat eine gewisse Zeitspanne zur Verfügung um sich mit der Verkehrssituation vertraut zu machen, bevor die Kontrolle vom autonomen System abgegeben wird.
  \item \emph{Full Self-Driving Automation:} Das Fahrzeug kann die komplette Kontrolle übernehmen ohne das eine Person in das Fahrgeschehen eingreifen können muss. Diese gibt nur noch das Ziel an.
\end{enumerate}

Die \ac{BAST} unterscheidet 3 Kategorien\cite{bast2021}:

\begin{itemize}
  \item \emph{Assistierter Modus:} Gleicht dem Level 1 der Definition der \ac{NHTSA}. In Einzelheiten kann ein automatisiertes System unterstützen, jedoch nie volle Kontrolle über das System ausüben.
  \item \emph{Automatisierter Modus:} Equivalent zu Level 2 der Definition der \ac{NHTSA}.
  \item \emph{Autonomer Modus:} Entspricht Level 5 der \ac{NHTSA} Definition.
\end{itemize}

Die Kategorien der \ac{BAST} wurden von der \ac{sae} weiterentwickelt \cite{bast2021} und unterscheidet sich im wesentlichen von der Definition der \ac{NHTSA} dadurch, dass zwischen Stufe 3 und 4 noch eine weitere hinzugefügt wurde, die es nur unter bestimmten Konditionen erlaubt die Kontrolle vollständig abzugeben\cite{SAE2021}.

Im Folgenden wird vor allem die Definition der \ac{BAST} relevant sein, da sie einfach, aber ausreichend ist.

\section{Datenquellen in modernen Fahrzeugen}

Moderne Fahrzeuge bestehen aus vielen Einzelsystemen, die über einen gemeinsamen Bus miteinander Kommunizieren.
Dabei fallen viele Daten an, die die forensische Auswertung unterstützen können. Im Folgenden werden die Systeme behandelt,
die für autonome Fahrzeuge eine relevante Funktion einnehmen.

% TODO: Relevanz der Systeme hervorheben

\subsection{Radar}

Radar Systeme sind eine relative alte Erfindung, finden aber auch in den modernsten Fahrzeugen noch Einsatz.
Dabei werden Radiowellen benutzt um die Entfernungen zu Objekten in der Umgebung zu messen.
Im Gegensatz zum Lidar werden Radiowellen deutlich weniger von der Umgebung verschluckt und haben damit eine
größere Reichweite.~\cite{Neal2018} Außerdem ist es möglich durch den Dopplereffekt die Bewegungsrichtung von Objekten abzuleiten.
Radare werden bereits für den assistierten Modus benutzt um z.B. Abstandsregeltempomaten umzusetzen.

\subsection{Lidar}

Lidar ist eine dem Radar ähnliche Technologie, die anhand der Zeit die ein Lichtimpuls braucht um zum Ziel und wieder zurückzukommen, berechnet wie weit das Ziel entfernt ist. Solche Systeme sind inzwischen so weit entwickelt, dass sie ein
dreidimensionales Abbild der Umgebung generieren können.
Dieses Abbild enthält allerdings ausschließlich Tiefeninformationen.~\cite{Liu2018}
Im Gegensatz zum Radar nimmt die Effektivität bei schlechten Wetterbedingungen, wie Schnee, Regen, Nebel und Staub, jedoch stark ab.~\cite{Neal2018}
Dieses System wird von den meisten Herstellern autonomer Fahrzeuge benutzt, jedoch nicht von allen.~\cite{Dickson2021}

\subsection{Kameras}

Neben Lidar/Radar Systemen kommen Kamerasysteme mit visueller Objekterkennung zum Einsatz.

\subsection{Entertainment System}

\subsection{Forensische Standards}

AVGuard + T-Box

\subsection{Schwierigkeiten der automotiven Forensik}

\section{Fazit}
\section*{Abkürzungen}
\addcontentsline{toc}{section}{Abkürzungen}

\begin{acronym}[IEEE]
  \acro{abs}[ABS]{Anti-lock braking system}
  \acro{BAST}[BASt]{Bundesanstalt für Straßenwesen}
  \acro{NHTSA}{National Highway Traffic Safety Administration}
  \acro{sae}[SAE]{SAE International}
\end{acronym}

% Literaturverzeichnis
\addcontentsline{toc}{section}{Literatur}
\printbibliography
\end{document}
