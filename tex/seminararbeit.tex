\documentclass[conference,compsoc,final,a4paper]{IEEEtran}
\usepackage[utf8]{inputenx}

%% Bitte legen Sie hier den Titel und den Autor der Arbeit fest
\newcommand{\autoren}[0]{Gleumes, Folke Henning}
\newcommand{\dokumententitel}[0]{Wird autonomes Fahren die forensische Untersuchung erschweren?}

% Hie muss normalerweise nichts angepasst werden
\usepackage[pdftex]{graphicx}
\graphicspath{{img/}}
\DeclareGraphicsExtensions{.pdf,.jpeg,.jpg,.png}
\usepackage[cmex10]{amsmath}
\usepackage{algorithmic}
\usepackage{array}
\usepackage{dblfloatfix}
\usepackage{url}
\usepackage[autostyle=true,german=quotes]{csquotes}
\usepackage[backend=biber,
            sorting=none,   % Keine Sortierung
            doi=true,       % DOI anzeigen
            isbn=false,     % ISBN nicht anzeigen
            url=true,       % URLs anzeigen
            maxnames=6,     % Ab 6 Autoren et al. verwenden
            minnames=1,     % und nur den ersten Autor angeben
            style=ieee,]{biblatex}
\usepackage{booktabs}
\usepackage{xcolor}
\usepackage{listings}             % Source Code listings
\usepackage[printonlyused]{acronym}
\usepackage{fancyvrb}
\usepackage{tocloft} % Schönere Inhaltsverzeichnisse

% Farben definieren
\definecolor{linkblue}{RGB}{0, 0, 100}
\definecolor{linkblack}{RGB}{0, 0, 0}
\definecolor{darkgreen}{RGB}{14, 144, 102}
\definecolor{darkblue}{RGB}{0,0,168}
\definecolor{darkred}{RGB}{128,0,0}
\definecolor{comment}{RGB}{63, 127, 95}
\definecolor{javadoccomment}{RGB}{63, 95, 191}
\definecolor{keyword}{RGB}{108, 0, 67}
\definecolor{type}{RGB}{0, 0, 0}
\definecolor{method}{RGB}{0, 0, 0}
\definecolor{variable}{RGB}{0, 0, 0}
\definecolor{literal}{RGB}{31,0, 255}
\definecolor{operator}{RGB}{0, 0, 0}

\usepackage[ngerman]{babel}

\DefineBibliographyStrings{ngerman}{
    andothers = {{et al\adddot}},  % Immer et al. sagen, auch bei Deutsch als Sprache
}
\usepackage[
      unicode=true,
      hypertexnames=false,
      colorlinks=true,
      colorlinks=false,
      linkcolor=darkblue,
      citecolor=darkblue,
      urlcolor=darkblue,
      pdftex
   ]{hyperref}
%	 \PrerenderUnicode{ü}


% Einstellungen für Quelltexte
\lstset{
    xleftmargin=0.1cm,
    basicstyle=\scriptsize\ttfamily,
    keywordstyle=\color{keyword},
    identifierstyle=\color{variable},
    commentstyle=\color{comment},
    stringstyle=\color{literal},
    tabsize=2,
    lineskip={2pt},
    columns=flexible,
    inputencoding=utf8,
    captionpos=b,
    breakautoindent=true,
    breakindent=2em,
    breaklines=true,
    prebreak=,
    postbreak=,
    numbers=none,
    numberstyle=\tiny,
    showspaces=false,      % Keine Leerzeichensymbole
    showtabs=false,        % Keine Tabsymbole
    showstringspaces=false,% Leerzeichen in Strings
    morecomment=[s][\color{javadoccomment}]{/**}{*/},
    literate={Ö}{{\"O}}1 {Ä}{{\"A}}1 {Ü}{{\"U}}1 {ß}{{\ss}}2 {ü}{{\"u}}1 {ä}{{\"a}}1 {ö}{{\"o}}1
}

\hypersetup{
    pdftitle={\dokumententitel},
    pdfauthor={\autoren},
    pdfdisplaydoctitle=true,
    hidelinks
}

% Makros für typographisch korrekte Abkürzungen
\newcommand{\zb}[0]{z.\,B.}
\newcommand{\dahe}[0]{d.\,h.}
\newcommand{\ua}[0]{u.\,a.}

% Wo liegt Sourcecode?
\newcommand{\srcloc}{src/}

% Literatur einbinden
\addbibresource{literatur.bib}
 % Weitere Einstellungen aus einer anderen Datei lesen

\begin{document}

% Titel des Dokuments
\title{\dokumententitel}

% Namen der Autoren
\author{
  \IEEEauthorblockN{\autoren}
  \IEEEauthorblockA{
    Hochschule Mannheim\\
    Fakultät für Informatik\\
    Paul-Wittsack-Str. 10,
    68163 Mannheim
    }
}

% Titel erzeugen
\maketitle
\thispagestyle{plain}
\pagestyle{plain}

\begin{abstract}
\end{abstract}

% Inhaltsverzeichnis erzeugen
{\small\tableofcontents}

\section{Einleitung}

Seit einigen Jahren befindet sich die Automobilindustrie im Umbruch. Nicht nur steht der Wechsel zu Strom betriebenen Kraftfahrzeugen an,
sondern auch der Wechsel zu immer mehr Computergestützten Fahrsystemen, bis hin zum komplett autonomen Fahrzeug.
Um diesen Umschwung zu ermöglichen, müssen immer mehr Daten erhoben werden und komplexere Systeme zum Auswerten dieser Daten geschaffen werden.
Nach Aussagen von Intel aus dem Jahr 2016 könnten Daten von bis zu 4 Terrabyte pro Tag generiert werden.~\cite{Nelson2016}
Die Firma Tuxera schätzte 2021 das ein durchschnittlicher US-amerikanischer Verbraucher zwischen 380 und 5100 TB pro Jahr generieren könnte.~\cite{Wright2021}
Gleichzeitig gibt es einige Initiativen, sowohl gesetzlich\cite{Boehm2020}, als auch technische\cite{Hoque_2021a}\cite{Lee_2019}, um die neu entstehenden Systeme der Forensik zugänglicher zu gestalten.
In dieser Arbeit soll die Frage geklärt werden, ob und wie diese Änderung die forensische Auswertung von Fahrzeugen verändern könnte.

\section{Automatisiertes/Autonomes Fahren}

% Eingehen auf relevante Punkte für die Forensik (z.B. Hände am Lenkrad)

Für den Begriff des autonomen oder automatisierten Fahrens gibt es mehrere Definitionen, die sich jedoch im Kern gleichen.
2013 definierte die \ac{NHTSA} folgende 5 Stufen\cite{NHTSA2013}:
\begin{enumerate}
  \setcounter{enumi}{-1}
  \item \emph{No-Automation:} Der Fahrer hat volle Kontrolle über das Fahrzeug. Dies gillt auch wenn das Fahrzeug über Warnsysteme, wie eine Kollisionswarnung verfügt. Sekundäre Systeme wie Scheibenwischer, Blinklichter, Beleuchtung gelten ebenfalls als Level 0.
  \item \emph{Function-specific Automation:} Die fahrende Person kann teilweise die Kontrolle über einzelne System dem Fahrzeug überlassen. Beispiele für solche Systeme sind das \ac{abs}, welches die ultimative Kontrolle beim Fahrer belässt, aber in den Bremsprozess eingreift und der Spurhalteassistent, welcher nur leicht in den Lenkprozess eingreift, aber jederzeit von der fahrenden Person überschrieben werden kann.
  \item \emph{Combined Function Automation:} Ab dieser Stufe können auch die primären Funktionen vollständig vom Fahrzeug übernommen werden, allerdings muss die fahrzeugführende Person jeder Zeit bereit sein in das Fahrgeschehen einzugreifen. Um dies sicherzustellen, gibt es zum Beispiel Distanzsensoren am Lenkrad, welche messen ob der Fahrer die Hände in der Nähe des Lenkrads hat. %TODO: quelle
  \item \emph{Limited Self-Driving Automation:} Der/Die Fahrer:in kann zeitweise, unter den korrekten Bedingungen, wie z.B. eine Autobahn bei guter Sichtbarkeit, die komplette Kontrolle über das System abgeben. Sollte sich ein Hindernis ankündigen, dass nicht von dem autonomen System übernommen werden kann, wird die fahrende Person benachrichtigt und hat eine gewisse Zeitspanne zur Verfügung um sich mit der Verkehrssituation vertraut zu machen, bevor die Kontrolle vom autonomen System abgegeben wird.
  \item \emph{Full Self-Driving Automation:} Das Fahrzeug kann die komplette Kontrolle übernehmen ohne das eine Person in das Fahrgeschehen eingreifen können muss. Diese gibt nur noch das Ziel an.
\end{enumerate}

Die \ac{BAST} unterscheidet 3 Kategorien\cite{bast2021}:

\begin{itemize}
  \item \emph{Assistierter Modus:} Gleicht dem Level 1 der Definition der \ac{NHTSA}. In Einzelheiten kann ein automatisiertes System unterstützen, jedoch nie volle Kontrolle über das System ausüben.
  \item \emph{Automatisierter Modus:} Equivalent zu Level 2 der Definition der \ac{NHTSA}.
  \item \emph{Autonomer Modus:} Entspricht Level 5 der \ac{NHTSA} Definition.
\end{itemize}

Die Kategorien der \ac{BAST} wurden von der \ac{sae} weiterentwickelt \cite{bast2021} und unterscheidet sich im wesentlichen von der Definition der \ac{NHTSA} dadurch, dass zwischen Stufe 3 und 4 noch eine weitere hinzugefügt wurde, die es nur unter bestimmten Konditionen erlaubt die Kontrolle vollständig abzugeben\cite{SAE2021}.

\section{Datenquellen in modernen Fahrzeugen}

Moderne Fahrzeuge bestehen aus vielen Einzelsystemen, die über einen gemeinsamen Bus miteinander Kommunizieren.
Dabei fallen viele Daten an, die die forensische Auswertung unterstützen können. Im Folgenden werden die Systeme behandelt,
die für autonome Fahrzeuge eine relevante Funktion einnehmen.

% TODO: Relevanz der Systeme hervorheben

\subsection{RADAR}

Radar Systeme sind eine relative alte Erfindung, finden aber auch in den modernsten Fahrzeugen noch Einsatz.
Dabei werden Radiowellen benutzt um die Entfernungen zu Objekten in der Umgebung zu messen.
Im Gegensatz zum Lidar werden Radiowellen deutlich weniger von der Umgebung verschluckt und haben damit eine
größere Reichweite.~\cite{Neal2018} Außerdem ist es möglich durch den Dopplereffekt die Bewegungsrichtung von Objekten abzuleiten.
Radare werden bereits für den assistierten Modus benutzt um z.B. Abstandsregeltempomaten umzusetzen.

\subsection{LiDAR}

\ac{LiDAR} ist eine dem Radar ähnliche Technologie, die anhand der Zeit die ein Lichtimpuls braucht um zum Ziel und wieder zurückzukommen, berechnet wie weit das Ziel entfernt ist.
Mit dieser Technik ist es möglich, zwei- und dreidimensionale Abbider der Umgebung zu schaffen.
Um dieses Ziel zu erreichen wurde in der Vergangenheit ein rotierender Spiegel genutzt, um den Laser auf einer Ebene um das Auto zu bewegen.
Dies erzeugt aber nur ein zweidimensionales Bild, also ein Abbild aller Objekte in einer Ebene um das Fahrzeug herum.
Um auch mechanisch ein dreidimensionales Bild erzeugen zu können wird zusätzlich die Neigung des Spiegels verändert, um aus der Ebene ausbrechen zu können.
Erkennbar sind diese Systeme an einem charakteristischen runden Aufbau auf dem Dach des Fahrzeuges.\\
Ein Ansatz, der weniger auf mechanische Bauteile setzt, benutzt einen starken Blitz im nicht sichtbaren Spektrum, der dann von einem zweidimensionalen Detektor aufgenommen werden kann.
Zusätzlich mit den Tiefeninformation, die aus der Verzögerung des Echos abgeleitet werden können, wird eine Punktwolke abgeleitet die ein dreidimensionales Abbild der Umgebung
schaffen. Im Gegensatz zum mechanischen \ac{LiDAR} hat diese Technik kaum bewegliche Teile und ist dadurch weniger verschleißanfällig, allerdings bringt der Lichtblitz ein eigenes Problem mit,
er führt zu starken Verbrauchsspitzen.~\cite{Zhaohua2020}\\
Im Gegensatz zum Radar nimmt die Effektivität bei schlechten Wetterbedingungen, wie Schnee, Regen, Nebel und Staub, bei beiden Ansätzen jedoch stark ab.~\cite{Neal2018}
Dieses System wird von den meisten Herstellern autonomer Fahrzeuge benutzt, jedoch nicht von allen.~\cite{Dickson2021}

\subsection{Kameras}

Die Kameras eines autonomen Vehikels sind wahrscheinlich die wichtigsten Sensoren am Fahrzeug.
Die Firma Tesla kündigte 2021 sogar an, nur noch auf Kameras zu setzen und das vorher noch benutzte Radar-System in Zukunft nicht mehr zu verbauen, sind damit aber ein Einzelfall.\cite{Koellner2022}
Kameras sind oft günstiger als andere Sensoren, haben allerdings die Schwäche das ihre Wahrnehmung, ähnlich wie beim LiDAR, leicht von schlechten Sichtverhältnissen beeinflusst werden.
Außerdem enthalten sie von sich aus keine Tiefeninformationen, die besonders für \ac{acc} und die Kollisionsvermeidung relevant sind.
Es ist allerdings möglich Tiefeninformationen aus Stereokameras abzuleiten, das benötigt allerdings zusätzlichen Rechenaufwand und führt zu Latenz.~\cite{Petit2022}\par

Eine weitere Anwendung finden die Kameras im Innenraum des Fahrzeuges.
Nach Einführung der Stufe 3 des autonomen Fahrens in Amerika hat sich in vielen Fällen gezeigt das die Fahrer
ihre Aufmerksamkeitspflicht verletzen und nicht in kurzer Zeit fähig sind in das Fahrgeschehen einzugreifen.
Mittels der internen Kameras kann überwacht werden, ob die Aufmerksamkeit des Fahrers tatsächlich der Straße gillt.
Zu diesem Zweck hatte Tesla Sensoren im Lenkrad verbaut, die die Präsenz der Hände am Lenkrad sicherstellen sollten,
diese stellten sich allerdings als unzureichend und einfach manipulierbar heraus.~\cite{Trudell2021}

\subsection{Ultraschall}

Ultraschallsensoren werden bereits für Einparkhilfen und Kollisionswarnungen genutzt.
Die Sensoren arbeiten vergleichbar wie das Radar, allerdings werden keine Radiowellen genutzt,
sondern kurzwelliger Schall, meistens zwischen 40kHz und 58kHz.~\cite{Zhaohua2020}
Diese Methode hat allerdings den Nachteil das die Reichweite auf ca. 10 Meter begrenzt ist, und somit nur
für Szenarien in denen sich das Fahrzeug langsam bewegt, geeignet ist.~\cite{Petit2022}

\subsection{Infotainment System}

Das Infotainment System ist eines der datenstärksten Systeme, die in einem Fahrzeug verbaut sind.
Wie dem Namen zu entnehmen ist, erfüllt dieses System zwei Funktionen:
Zum einen macht es dem Fahrer Informationen über das Auto zugänglich zu und zum anderen kann es zur Unterhaltung genutzt werden.

Während diese Systeme lange nur als Radio und zum Abspielen eigener physischer Medien wie Audiokassetten und CDs genutzt wurden,
sind etwa ab 2010 viele neue Funktionen dazu gekommen.~\cite{Rossi2021}\par

Mit der Einführung von Bluetooth ist es nicht nur möglich geworden Audio über die integrierte Audioanlage des Autos abzuspielen,
sondern diese auch für Anrufe zu benutzen. Durch das \ac{AVRCP} Protokoll werden zusätzlich Metadaten über die wiedergegebenen Medien
an das Infotainment System gesendet, die bei einer forensichen Untersuchung relevant sein könnten.
Ebenfalls ist es möglich sein Telefonbuch in das Infotainment System hochzuladen, ein Feature, das es dem Nutzer erleichtern soll
Anrufe initieren zu können. Einige Modelle bieten Sprachsteuerungen an, mit denen der Fahrer einen Anruf starten kann, ohne den Blick von
der Straße zu nehmen.\par
Bei der Verwendung von der Bluetoothschnittstelle hinterlässt das verbunde Gerät einige Informationen, die zur eindeutigen Identifikation
genutzt werden können und damit sehr wertvoll für eine Investigation des Fahrzeuges sind, sollte es nötig sein den Fahrer zu identifizieren.

Ein weiteres weit verbreitetes Feature ist die Integration eines Navigationssystems.
Zur Funktion ist ein \ac{GPS} notwendig, womit die Position des Fahrzeuges auf wenige Meter genau festgestellt werden kann.
% TODO: GPS 

Daten die bei bisherigen Untersuchungen oft gefunden wurden:~\cite{Lacroix2017}
\begin{itemize}
  \item \emph{Bluetooth Geräteliste} Die Geräteliste enthält alle verbundenen Geräte. Unter anderem auch der Mac Addresse, die zur eindeutigen Identifikation eines Gerätes genutzt werden kann
  \item \emph{Kontakte} Die per Bluetooth hochgeladene Liste der Kontakte
  \item \emph{Anruf Verlauf} Liste getätigter Anrufe über das Infotainment System
  \item \emph{SMS} Empfangene SMS während ein Telefon verbunden war
  \item \emph{GPS Koordinaten} Gespeicherte Orte für die Navigation, teilweise auch Orte an denen sich das Fahrzeug befunden hat
\end{itemize}

% Bluetooth
% gps
% Wlan
% nutzer interaktion

\section{Schwierigkeiten der automotiven Forensik}

Die folgende Sektion basiert größtenteils auf den Erkenntnissen aus \citetitle{Kopencova_2020} und \citetitle{LeKhac2020}

% Fahrzeuge werden komplexer

% Anzahl der Systeme wächst

\section{Gesetzliche Initiativen}



% Jackpot: DSSAD / EDR

\section{Forensische Standards}

AVGuard + T-Box

\section{Fazit}
\section*{Abkürzungen}
\addcontentsline{toc}{section}{Abkürzungen}

\begin{acronym}[IEEE]
  \acro{GPS}{Global Positioning System}
  \acro{AVRCP}{Audio/Video Remote Control}
  \acro{abs}[ABS]{Anti-lock braking system}
  \acro{BAST}[BASt]{Bundesanstalt für Straßenwesen}
  \acro{NHTSA}{National Highway Traffic Safety Administration}
  \acro{sae}[SAE]{SAE International}
  \acro{acc}[ACC]{Adaptive Cruise Control}
  \acro{LiDAR}{Light detection and ranging oder Light imaging, detection and ranging}
\end{acronym}

% Literaturverzeichnis
\addcontentsline{toc}{section}{Literatur}
\printbibliography
\end{document}
